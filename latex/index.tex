HSENCFS (High Security En\+Crypting File System)

High security encryption file system. We make use of the API offered by the fuse subsystem to intercept file operations.

The interception is done between mountsecret and mountpoint. Copying data to mountpoint ends up encrypted in mountsecret. Copying data from mountpoint is sourced from mountsecret and decrypted. See \char`\"{}hsencrw.\+c\char`\"{}.

Use a dotted file for mountsecret (like .data or .secretdata)

One additional useful feature is auditing. Reports file access by user ID. The report is sent to syslog. (use the -\/l option to turn on log)

To make it, type\+: \begin{DoxyVerb}make
.. or gcc -lfuse -lulockmgr [localobjects ...] hsencfs.c -o hsencfs
\end{DoxyVerb}
 To use it\+: \begin{DoxyVerb} hsenc [-f] ~/secrets  ~/.secrets
\end{DoxyVerb}
 Command above will expose the $\sim$/secrets directory. It is sourced from the backing directory $\sim$/.secrets 